\documentclass[11pt]{article}
\usepackage[letterpaper,margin=1in]{geometry}
\usepackage{unicode-math}
\setmainfont{TeX Gyre Termes}
\setmathfont{TeX Gyre Termes Math}
\usepackage{linguex}
\usepackage{hyperref}
\setlength{\parindent}{0pt}
\setlength{\parskip}{1ex}
\usepackage{enumitem}
\setlist{left=0pt}
\title{Scope Economy Task Design}
\author{Yurika, Berg\"ul, Taieba, Haoming}


\begin{document}
\maketitle
\section{The questions that the experiment will address}
\label{sec:the_questions_that_the_experiment_will_address}


\begin{itemize}
  \item Is children's LF movement sensitive to the combination of the locality
    and economy conditions observed in adults? Ultimately, is LF movement
    acquired as being inherently subject to the locality and economy
    conditions?
  \item Specifically, is children's LF movement also sensitive to Scope
    Economy and Shortest Move as defined based on the adult data?
\end{itemize}

\section{Task design}
\label{sec:task_design}

\begin{itemize}

  \item A story that is compatible only with the inverse scope reading is presented to the child.

  \item The child is then asked to give truth value judgements to the sentences.

  \item The surface and inverse scope readings of the sentences are listed after the SS and IS labels, respectively.

  \item Scope Economy predicts that the inverse scope reading for \ref{ex:defa} and \ref{ex:aa} are unavailable.

  \item The expected truth value judgements (assuming children have adult-like behavior) are specified in the parenthesis.

  \item The test items are as follows:
    \ex. \label{ex:ea} A monkey hugged every rabbit, and a dog did, too. (T)
    \a.[SS] \(∃\) \(∀\) and \(∃\) \(∀\) (F)
    \b.[IS] \(∀\) \(∃\) and \(∀\) \(∃\) (T)

  \item Because \ref{ex:ea} allows both readings and one of them is compatible with the story, it will be judged true.

    \ex. \label{ex:defa} A monkey hugged every rabbit, and Cookie Monster did, too. (F)
    \a.[SS]   \(∃\) \(∀\) and Def \(∀\) (F)
    \b.[IS] *\(∀\) \(∃\) and Def \(∀\) 

  \item The reading of \ref{ex:defa} that is compatible with the story is ruled out by Scope Economy, and so will be judged false.

    \ex. \label{ex:aa} A monkey hugged every rabbit, and every dog did, too. (F)
    \a.[SS] \(∃\) \(∀\) and \(∀\) \(∀\) (F)
    \b.[IS] *\(∀\) \(∃\) and \(∀\) \(∀\) 

  \item The reading of \ref{ex:aa} that is compatible with the story is ruled out by Scope Economy, and so will be judged false.

\end{itemize}
% Second conjunct scopally ambiguous
% (4) Cookie Monster hugged every rabbit, and a monkey did, too. (F)
% Def ∀ and ∃ ∀
% *Def ∀ and ∀∃ (story)
%
% \item
%   Every monkey hugged every rabbit, a monkey did, too. (F)
%   ∀∀ and ∃∀
%   *∀∀ and ∀∃ (story)


\section{Control ideas}
\label{sec:control_ideas}


\begin{itemize}
  \item No VP ellipsis and simple sentences with inverse scope 
    \ex. A monkey hugged every rabbit.

  \item No VP ellipsis and conjunction of inverse scope and surface scope 
    \ex. A monkey hugged every rabbit and Cookie Monster hugged every dog.

\end{itemize}

% feel free to have multiple semi-defined task/design ideas, we can decide
% in class)
% it's always good to have at least 1 item per condition, for concreteness
%
% different idealized outcomes -\/- e.g.~draw some graphs on the board
% representing different outcomes

\section{Linking hypothesis}
\begin{itemize}
  \item Truth value judgment reflects available LF structures for the sentence.
\end{itemize}

\section{Possible patterns of results}

\begin{enumerate}[label=(\alph*)]
  \item \label{itm:control} If \ref{ex:ea} is judged true with inverse scope in inverse scope-only
    context: kids have inverse scope.
  \item If \ref{itm:control} holds and 
    \begin{itemize}
      \item \ref{ex:defa} and \ref{ex:aa} are judged false: kids have Scope Economy and Shortest Move;
        % \item \ref{ex:defa} and \ref{ex:aa} are judged true but (4) and (5) are
        %   judged false, kids have Scope Economy+Shortest Move but have
        %   processing difficulty when they need to hold two scope possibilities
        %   in working memory.
      \item \ref{ex:defa} and \ref{ex:aa} are judged true: kids lack at least one of Scope
        Economy and Shortest Move.
    \end{itemize}

\end{enumerate}

\section{Questions for class}

\begin{itemize}
  \item Should we have a conjunction of two clauses or two independent clauses?
    \ex. \a. A monkey hugged every rabbit, and a dog did, too.
    \b. A monkey hugged every rabbit. A dog did, too.

  \item How to block so that we make sure that all kids have inverse scope in the absence of Scope Economy effects but don't get carryover effect?
  \item Acting out or one still slide?
\end{itemize}

% Same number of monkeys and dogs as rabbits!
% Verbs: hug, pet,

\end{document}
